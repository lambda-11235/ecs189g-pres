\documentclass[12pt]{article}
\usepackage{amsmath}
\usepackage{amssymb}
%\usepackage{asymptote}
\usepackage{calc}
%\usepackage{caption}
%\usepackage{chemfig}
\usepackage{color}
\usepackage{commath}
%\usepackage{enumitem}
%\usepackage[c]{esvect}
%\usepackage{etoolbox}
\usepackage{fancyhdr}
%\usepackage{float}
%\usepackage{fontspec}    %fontspec only works with xetex and luatex.
%\usepackage{fp}
\usepackage{geometry}
%\usepackage{graphicx}
\usepackage{lastpage}
%\usepackage{listings}
%\usepackage{luacode}
\usepackage{mathtools}
%\usepackage{mhchem}
%\usepackage{pgfplots}
%\usepackage{setspace}
\usepackage{siunitx}
%\usepackage{tcolorbox}
%\usepackage{tikz}
\usepackage{todonotes}

\geometry{hmargin=2cm,vmargin=4cm}

% For multipage align*
\allowdisplaybreaks

\title{Script}
\author{Taran Lynn}
\date{}

\begin{document}
\maketitle

\begin{enumerate}
\item Start off by asking what the purpose of data visualization is.
  \begin{enumerate}
  \item The purpose is to see how data is structured.

  \item This is done through manifold learning.
    \todo{what is a manifold, how do we learn?}
    [ Show example of 2D manifold in 3D space being put on a 2D space ]
  \end{enumerate}
  
\item Broad overview of how Stochastic Network Embedding works.
  \begin{enumerate}
  \item SNE tries to preserve the higher dimensional distances
    between point on a 2D or 3D plain.
    
  \item If we imagine that each data pair of points in higher
    dimensional space is connected by a spring of length $p_{j|i}$.
    SNE can be thought of as trying to squash points down into lower
    dimensional space.
    The spring lengths will $q_{j|i}$ probably be different, but their
    force, caused by gradient descent, will make them to approach
    $p_{j|i}$.
    Not a completely accurate analogy because $p_{j|i} \neq p_{i|j}$,
    so spring length is different when viewed from $x_i$ and $x_j$
    (important later).
    [ Show spring graphic ]

  \item First, each data point is assigned a similarity $p_{j|i}$ to
    every other point based on where it is in a higher-dimensional
    Gaussian probability density centered on $x_i$.
    Ignore how $\sigma_i$ is chosen for now.
    [ Use 2D Gaussian as an example ]

  \item Then we come with create a set of lower dimensional points
    equal in number to the higher dimensional ones.
    These point get similarity values $q_{j|i}$.
    We then perform gradient descent to minimize the difference
    between every $p_{j|i}$ and $q_{j|i}$ for every $i$ and $j$, as
    measured by the Kullbeck-Leibler divergence.
    [ Note that since $P_i$ and $Q_i$ are probability distributions, KL is $\ge 0$ ]

  \item $\sigma_i$ is chosen via binary search on perplexity.
    Can be interpreted as a smooth measure the number of neighbors.
    [ $H(P_i)$ is the entropy in the neighborhood sample, i.e. the
    number of bits needed to represent the uncertainty, so $2^{H(P_i)}$ is
    like the number of neighbors. ]
  \end{enumerate}

\item \todo[inline]{Problems with SNE}
\end{enumerate}

\end{document}